% !Mode:: "TeX:UTF-8" 

\BiChapter{绪论}{Introduction }

本章在第一部分介绍了可搜索加密的研究背景及意义,在第二部分对对称可搜索加密的研究现状进行了阐述,
在第三部分介绍了本文的工作及贡献,在第四部分罗列出了本文的组织结构。



%=========================================================================================
\BiSection{研究背景及意义}{Research Background and Significance}
随着云计算和大数据的发展,越来越多的用户数据被存放在云端。这些数据中通常包含一些敏感数据,如个人身份证号,健康记录等等。
如果数据以明文形式存放在云端,则很容易导致隐私信息的泄露。
比如, 2013 年亚马逊的云平台受到黑客攻击,导致 2 亿 5 千 9 百万条注册会员的个人信息被泄露。
2017 年和美国某党派合作的数据分析商 Deep Root Analytics 以及 Data Trust 放在亚马逊的 1.1 TB的数据发生泄露。
因此为了保护用户的隐私数据,会在将数据上传到云服务器之前以某种加密方式对其进行加密。
按照传统方式,为了使用云端的密文数据,通常会先用大量的网络带宽将密文下载,在本地完全解密后再搜索。
然而这种方案有以下两个致命缺点:
(1) 如果云服务器上含有大量数据,一一下载会占用大量网络带宽。 
(2) 对已下载的文件完全解密会占用大量本地计算资源,效率极低。解决此类问题的技术称为可搜索加密(Searchable Encryption,SE),
在该技术中,数据用户对感兴趣的关键词进行加密后产生陷门,云服务器利用陷门搜索和关键词相关的数据。
在可搜索加密方案中,安全性和搜索效率是最为关注的两个问题。
方案满足的安全模型越强,抵抗攻击的能力就越强,用户数据泄露的可能性就越低。
搜索效率越高,用户的等待时间就越短,用户体验就越好。
因此,设计出高效且满足强安全模型的可搜索加密方案具有非常重要的意义。


%-----------------------------------------------------------------------------------------
\BiSection{国内外研究现状}{Domestic and Overseas Research at Present}
可搜索加密技术分为两大类:对称可搜索加密(Symmetric Searchable Encryption, SSE)和
非对称可搜索加密(Asymmetric Searchable Encryption, ASE)。
非对称可搜索加密又被称作公钥可搜索加密(Public key Encryption with Keyword Search, PEKS)。
两类技术在构造方法和使用场景方面都有不同,具体来讲:(1) 两类技术的构造方法不同。
其中,SSE方案的构造基于对称密码原语来设计。一般有两方参与者:数据拥有者和云服务器。
数据拥有者用私钥加密数据后外包给云服务器,之后数据拥有者使用私钥搜索云服务器中的数据。
与之对应的PEKS方案的构造基于公钥密码原语。一般有三方参与者:数据拥有者、云服务器和用户。
数据拥有者用公钥加密数据后外包给云服务器,之后用户使用私钥搜索云服务器中的数据。由于PEKS方案基于公钥密码构造,因此方案的构造效率低下。
(2) 两类方法的应用场景不同。SSE的应用场景多样,如财务数据、医疗数据、政府数据等私有数据库的外包与搜索,
与之对应的PEKS主要应用于加密邮件系统。基于上述两个原因,SSE技术成为近年来学术界的研究热点,大量SSE方案被提出。

目前, SSE 方案的研究内容主要分为如下四个方面:
(1) 单关键词搜索:单关键词搜索是 SSE 的基本搜索方式,即返回包含单个搜索关键词的所有文档。
在单关键词搜索方面,如何提高搜索效率和安全性成为研究热点;
(2) 多模式搜索:为了构造更加实用的 SSE 方案,需要支持更加丰富搜索的方式,如多关键词搜索、模糊关键词搜索、搜索结果排序和范围查询等;
(3) 前/后向安全搜索:动态 SSE 方案需要支持文件的更新和删除,但是在动态更新中会泄漏重要信息,如何保证动态 SSE 方案的前/后向安全成为关键;
(4) 可验证搜索:在恶意服务器模型中,如何设计可验证的 SSE 方案,即验证服务器返回结果的正确性和完整性成为必要。
接下来主要从这4个方面对 SSE 的研究现状进行阐述。

\BiSubsection{单关键词搜索}{Single Keyword Search}
2000年,Song\citeup{Song}等人首次提出了不可信赖服务器的存储问题,同时提出了第一个基于密文扫描思想的对称可搜索加密方案。
其方案在加密时将明文文件划分为“单词”,然后对其分别加密,搜索时通过对整个密文文件和密文单词进行比对,就可确认关键词是否存在,
甚至统计其出现的次数。其缺点也很明显,首先必须使用固定大小的“单词”,即通常要对文中的单词进行填充到固定长度变为“单词”,
此外,搜索时搜索效率较低,需要对密文进行全文扫描。
2003年,Goh\citeup{Goh_forward_index}首次提出了安全索引的概念,并提出了第一个安全模型——针对选择关键词攻击下的
语义安全(semantic security against adaptive chosen keyword attack, IND-CKA),
并且也给出了一个基于布隆过滤器\citeup{bloom_filter}(Bloom Filter, BF)且满足IND-CKA安全模型的Z-IDX方案。
其方案使用布隆过滤器作为单个文件的索引结构,将文件包含的关键词映射为码字存储于该文件的索引中,
通过布隆过滤器的运算,就能判定密文文件是否包含某个特定关键词,较全文扫描思想方案而言极大的提升了搜索效率。
其缺点来源于布隆过滤器中存在的误判问题。
% 2004年,Boneh\citeup{Boneh_rsa_based}等人提出了不可信赖服务器的路由问题:
% Bob 通过不可信赖邮件服务器向 Alice 发送包含某些关键词的邮件,要求服务器不能获取邮件内容和相关关键词信息,
% 但需根据关键词将邮件路由至 Alice 的某个终端设备。例如,如果邮件的关键词为“urgent”,则服务器将邮件分配至 Alice 的手机,
% 如果邮件的关键词为“office”,则服务器将邮件分配至 Alice 办公室的电脑。
% 为了解决该问题,Boneh等人提出了非对称可搜索加密的概念——支持关键词搜索的
% 公钥加密(public key encryption with keyword search, PEKS)概念。
% 他们基于 BF-IBE\citeup{Boneh_ref_himself} 构造了第一个满足双线性 Diffle-Hellman(bilinear Diffie-Hellman, BDH)
% 数学假设的 BDOP-PEKS 方案。
2005年,Chang\citeup{Chang_invert_index}等人考虑了可搜索加密基本问题的一个应用场景:用户通过个人电脑将明文文件加密后存放至云服务器,
然后使用移动设备(例如手机等)搜索服务器上的密文文件,
并针对此问题了提出了 PPSED(privacy preserving keyword searches on remoted encrypted data)方案。
其方案利用关键词字典集合给文件建立索引,索引的大小和关键词字典的数量相同。
该方案可以实现对文件的精确搜索,解决了布隆过滤器方案当中存在的误判问题。
2006年,Curtmola\citeup{Security_model}规范化了对称可搜索加密及其安全目标,
同时提出了能在非自适应和自适应安全模型下达到不可区分性安全的 SSE-1 和 SSE-2 方案。
类似于倒排索引的构建方式,SSE-1 和 SSE-2 都是基于“关键词-文件”索引构建思想,服务器只需 O(1)时间即可完成搜索操作。
然而,执行文件的添加或删除操作需要重新构建索引,时间开销较大。

\BiSubsection{多模式搜索}{Multi-modal Search}
仅仅使用单关键词搜索会使得服务器返回大量满足搜索条件的文档,为了进一步筛选出想要的文档,
支持多模式的SSE方案引起人们的关注,比如连接关键词搜索(同时搜索多个关键词)、模糊关键词搜索(同时搜索输入关键词的同义词)、
排序搜索(对查询结果进行排序,返回top-K的结果)和范围查询(对数值型数据的区间查询,如年龄位于18岁以上的)等。

2004年,Golle\citeup{first_multikeyword_scheme}等人第一次提出了 2 个支持多个连接关键词搜索的方案 GSW-1 和 GSW-2。
GSW-1方案的缺点是关键词的陷门大小与加密文档的数量成线性关系,
GSW-2方案利用双线性映射实现了常量大小的关键词陷门,但是判断一个文档需要计算两次双线性对。
2005年,Ballard\citeup{second_multikeyword_scheme}等人也构造了两个基于连接关键词的可搜索加密方案 SCKS-SS 和 SCKS-XDH,
其缺点和Golle等人提出的方案相同,都有大量的模指数运算和双线对运算,并且搜索复杂度都与文档数量线性相关。因此,其搜索效率不高。
Byun 等人\citeup{byun_multikeyword}和 Ryu 等人\citeup{ryu_multikeyword}分别利用双线性对
构造了基于连接关键词的可搜索加密方案 BLL 和 RT,方案的特点都是关键词陷门大小固定,但是判断每个文档都需要计算两次双线性对。
2013年,Cash等人\citeup{Cash_sublinear_multikeyword}提出了首个亚线性的连接关键词查询方案(Oblivious Cross-Tags, OXT)。
其方案主要思想分为两步: 首先,选取频率最小的关键词(在所有搜索的关键词中,频率最小的关键词对应的文档数最少)来搜索;
然后,判断搜索结果是否包含其它搜索的关键词,若该结果包含其它所有的搜索关键词,则该结果是最终的搜索结果。
Lai\citeup{Lai_modify_cash_hide_result}指出Cash的方案存在结果模式泄露问题:
即对于每个文档,知道其是否包含除了最小频率关键词外的其他所有关键词的信息。
为了避免这种不必要的泄露,他们提出了向量隐藏加密技术(Hidden Vector Encryption,HVE),并基于该技术提出结果模式隐藏的连接关键词搜索方案HXT。
传统的SSE方案只允许数据拥有者自身去执行搜索,为了支持多用户场景,Sun等人\citeup{Sun_OXTbased_multi-user_fine-grained}
基于OXT提出了无交互的细粒度多用户SSE方案。在Sun等人提出的方案中,不同的授权用户拥有不同的搜索和解密权限:
搜索权限为每个用户被授予不同的搜索关键词集合,用户只能搜索被授权的关键词;
解密权限为文档标识由属性基(Attribute-Based Encryption,ABE)加密,只有用户的属性满足加密策略时,才可以解密文档标识。
Wang等人\citeup{Wang_fix_sun_reduce_communication_cost}指出该多用户搜索方案有严重的通信和计算代价。
主要原因为:文档标识由ABE加密,服务器无法判断用户是否能解密某个加密文档。因此,服务器需要返回所有满足搜索条件的结果,
用户需自行判断能否解密,带来了额外的通信和解密代价。为此,Wang等人提出了服务器端匹配的匿名属性ABE技术,在不泄漏用户属性的情况下,
服务器可以判断用户能否解密密文。用此技术加密文档标识,服务器可以仅仅返回满足搜索条件且用户能够解密的结果,降低了通信代价并提高了解密效率。
2017年,Kamara等人\citeup{Kamara_OXTbased_disjunctive_multikeyword}指出OXT在执行析取关键字搜索(返回包含任意查询关键词的文档)时效率低下的问题,
并在OXT的基础上实现高效的析取关键字搜索。

2010年,Li等人\citeup{Li_first_fuzzy_keyword}首次提出了模糊关键词检索的方案。
该方案中,采用编辑距离来定义和度量关键词间的相似度,并使用了基于通配符和基于克(gram)的两种模糊关键词集构造方法。
关键词查询时,用户计算待查询关键词在编辑距离门限下的模糊陷门集合并交给服务器,服务器使用陷门集与存储的模糊关键词集进行一一匹配,
返回可能包含被查询关键词的密文文件集合。
2012 年,Wang 等人 [14] 进一步研究模糊关键词检索方案,并给出了形式化的安全性证明。
2012年,Kuzu等人\citeup{Kuzu_second_similarity_search}提出了一个可以在大规模加密数据集上进行相似性查询的安全索引方案。
它的基本思想是:利用局部敏感哈希对数据进行哈希操作,并使用一个向量记录该数据项,由哈希结果和对应向量组成哈希桶,其中哈希结果用作桶标记。
为了满足数据安全性的要求,对桶标记和向量分别进行加密,所有的加密哈希桶形成一个安全索引,再将加密的原始数据和安全索引存储在云服务器上供合法用户使用。
查询时先对关键字进行同样的哈希操作,然后将哈希结果和索引中的哈希桶一一比较。
最后分析、计算碰撞成功的哈希桶中的向量,返回碰撞次数符合要求的数据项。

%=========================================================================================
\BiSection{怎样用 \LaTeX}{How} 

本模板在 Windows + TeXLive2016 + Texsdudio 平台下开发,采用 XeLaTex 编译。虽然之前也开发过一个基于 CTeX 的模板,但是经过多方面比较发现 TeXLive+XeLaTex 处理中文更好,所以基于 CTeX 的模板没有共享。

{\color{red}本模板不能在 CTeX 软件下使用,必须采用 TeXLive,并且编译方式是 XeLaTeX。TeXLive 每年更新一个版本,我用的是 TeXLive2016。文本编辑器可以根据自己的喜好选用,我用的是 Texsdudio,这款开源软件非常不错,推荐大家使用。}

本模板的源文件通过主目录下的 main.tex 统一管理,setup 文件夹中存放格式定义和封面、摘要、目录等内容,body 文件夹中存放论文正文章节的源文件,appendix 文件夹中存放附录、致谢和声明等内容。

本模板只提供论文的格式定义,不提供 \LaTeX{} 的详细使用方法。%所以只回复和论文格式相关的问题,不解答具体的排版方法和技巧。
因为 \LaTeX{} 的资源非常丰富,大家可以在网上查找资料和并参与讨论,这样学习效率更高。我关注的两个网站是:\url{http://bbs.ctex.org/forum.php} 和 \url{http://www.latexstudio.net};参考的两本书是 ``The Not So Short Introduction to \LaTeXe'' 和 ``LaTeX2e完全学习手册''。

\BiSection{实际显示内容}{这里对应显示目录的内容} 

crap