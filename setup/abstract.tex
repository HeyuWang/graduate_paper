% !Mode:: "TeX:UTF-8" 

%====================================== 中文摘要 ==========================================
\BiAppendixChapter{摘~~~~要}{ABSTRACT (Chinese)}
\setcounter{page}{1}\pagenumbering{Roman}
\defaultfont

随着云计算和大数据的发展,越来越多的用户数据被存放在云端。这些数据中通常包含一些敏感数据,如个人身份证号,健康记录等等。
如果数据以明文形式存放在云端,则很容易导致隐私信息的泄露。因此为了保护用户的隐私数据,会在将数据上传到云服务器之前对其进行加密。
为了使用云端的密文数据,通常会先用大量的网络带宽将密文下载,在本地完全解密后再检索。然而这种方案有以下两个致命缺点:
1.如果云服务器上含有大量数据,一一下载会占用大量网络带宽。
2.对已下载的文件完全解密会占用大量本地计算资源,效率极低。
解决此类问题的技术称为可搜索加密(Searchable Encryption,SE),该技术要求只有被授权的用户才具有检索能力。

本文提出并实现了一种新的对称可搜索加密方案,并将其和已有的方案在实验上进行了对比。
同数据库使用索引提高搜索效率一样,对称可搜索加密方案通常会针对明文文件集构建相应的安全索引(不泄露相应明文文件集合信息的索引)。
本文基于一种新型的数据结构——不可区分布隆过滤器(Indistinguishable Bloom Filter,IBF1)\cite{adaptive},提出了一种新的安全索引构建算法。
关键词在被送往云服务器进行搜索前,需要通过某种变换来隐藏关键词中的明文信息,同时使得变换后的词又能够在安全索引上进行匹配搜索。
这种变换被称为陷门(trapdoor)生成算法,本文也提出了相应的陷门生成算法和陷门在安全索引上的搜索算法。
为了验证方案的有效性,分别用C语言实现了我们提出的方案以及另一种基于IBF的可搜索加密方案。从
1.安全索引构建时间,
2.安全索引占用的磁盘空间,
3.搜索安全索引花费的时间
这三个方面对两种方案进行了对比。
在真实数据集上的实验结果表明,本文提出的方案极大的减少了安全索引的构建时间和对磁盘的使用空间情况,同时搜索时间和另一方案处于同一数量级(ms量级)。
同时本文提出的方案也满足目前提出的最强的安全模型——自适应安全模型。

安全索引占据的磁盘空间以及搜索时间3个方面进行了对比。
本文第一次提出了使用作为树节点来构造一棵
变长不可区分布隆过滤器树(Variable-length Indistinguishable Bloomfilter Tree,VBTree)的方式来构建安全索引,
同时也给出了与VBTree相对应的陷门(trapdoor)算法以及针对相应陷门的服务器端的搜索算法。

% 博士学位论文摘要正文为 1000 字左右。

% 内容一般包括:从事这项研究工作的目的和意义;完成的工作 (作者独立进行的研究工作及相应结果的概括性叙述);获得的主要结论 (这是摘要的中心内容)。博士学位论文摘要应突出论文的创新点。

% 摘要中一般不用图、表、化学结构式、非公知公用的符号和术语。

% 如果论文的主体工作得到了有关基金资助,应在摘要第一页的页脚处标注:本研究得到某某基金 (编号:) 资助。


\vspace{\baselineskip}
\noindent{\fontsize{11.5pt}{11.5pt}\selectfont\bfseries 关\hspace{0.5em}键\hspace{0.5em}词}:对称可搜索加密, IBF, VIBtree

\vspace{\baselineskip}
\noindent{\fontsize{11.5pt}{11.5pt}\selectfont\bfseries 论文类型}:应用基础

\clearpage

%====================================== 英文摘要 ==========================================
\BiAppendixChapter{ABSTRACT}{ABSTRACT (English)}

\noindent 英文摘要正文每段开头不缩进,每段之间空一行。\\

\noindent The abstract goes here. \newline

\noindent \LaTeX{} is a typesetting system that is very suitable for producing scientific and mathematical documents of high typographical quality.

%\noindent You will never want to use Word when you have learned how to use \LaTeX.

\vspace{\baselineskip}
\noindent{\fontsize{11.5pt}{11.5pt}\selectfont\bfseries KEY WORDS}: Xi'an Jiaotong University, Doctoral dissertation, \LaTeX{} template

\vspace{\baselineskip}
\noindent{\fontsize{11.5pt}{11.5pt}\selectfont\bfseries TYPE OF DISSERTATION}: Application Fundamentals

\clearpage
